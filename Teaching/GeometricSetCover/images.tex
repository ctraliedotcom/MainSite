\batchmode
\documentclass[16pt]{article}
\RequirePackage{ifthen}




\usepackage{fullpage}
\usepackage{mathptmx}
\usepackage{latexsym}
\usepackage{graphicx}
\usepackage{amsmath,amssymb,amsthm,amsfonts}
\usepackage{algpseudocode}

%
\providecommand{\handout}[5]{
   \renewcommand{\thepage}{#1-\arabic{page}}
   \noindent
   \begin{center}
   \framebox{
      \vbox{
    \hbox to 5.78in { {\bf COMPSCI 634: Geometric Algorithms} \hfill #2 }
       \vspace{4mm}
       \hbox to 5.78in { {\Large \hfill #5  \hfill} }
       \vspace{2mm}
       \hbox to 5.78in { {\it #3 \hfill #4} }
      }
   }
   \end{center}
   \vspace*{4mm}
} 


%
\renewcommand{\paragraph}[1]{\medskip \noindent {\bf #1.}}

%
\providecommand{\lecture}[4]{
   \renewcommand{\thepage}{#1-\arabic{page}}
   \noindent
   \begin{center}
   \framebox{
      \vbox{
    \hbox to 5.78in { {\bf COMPSCI 634: Geometric Algorithms} \hfill #2 }
       \vspace{4mm}
       \hbox to 5.78in { {\Large \hfill Lecture #1  \hfill} }
       \vspace{2mm}
       \hbox to 5.78in { {\it Lecturer: #3 \hfill Scribe: #4} }
      }
   }
   \end{center}
   \vspace*{4mm}
} 



\newtheorem{definition}{Definition} 
\newtheorem{algorithm}{Algorithm} 
\newtheorem{openquestion}{Open Question} 
\newtheorem{remark}{Remark} 
\newtheorem{theorem}{Theorem} 
\newtheorem{lemma}[theorem]{Lemma} 
\newtheorem{corollary}[theorem]{Corollary} 
\newtheorem{fact}[theorem]{Fact} 
%
\providecommand{\example}[1]{\paragraph {Example #1}} 




\usepackage[dvips]{color}


\pagecolor[gray]{.7}

\usepackage[]{inputenc}



\makeatletter

\makeatletter
\count@=\the\catcode`\_ \catcode`\_=8 
\newenvironment{tex2html_wrap}{}{}%
\catcode`\<=12\catcode`\_=\count@
\newcommand{\providedcommand}[1]{\expandafter\providecommand\csname #1\endcsname}%
\newcommand{\renewedcommand}[1]{\expandafter\providecommand\csname #1\endcsname{}%
  \expandafter\renewcommand\csname #1\endcsname}%
\newcommand{\newedenvironment}[1]{\newenvironment{#1}{}{}\renewenvironment{#1}}%
\let\newedcommand\renewedcommand
\let\renewedenvironment\newedenvironment
\makeatother
\let\mathon=$
\let\mathoff=$
\ifx\AtBeginDocument\undefined \newcommand{\AtBeginDocument}[1]{}\fi
\newbox\sizebox
\setlength{\hoffset}{0pt}\setlength{\voffset}{0pt}
\addtolength{\textheight}{\footskip}\setlength{\footskip}{0pt}
\addtolength{\textheight}{\topmargin}\setlength{\topmargin}{0pt}
\addtolength{\textheight}{\headheight}\setlength{\headheight}{0pt}
\addtolength{\textheight}{\headsep}\setlength{\headsep}{0pt}
\setlength{\textwidth}{349pt}
\newwrite\lthtmlwrite
\makeatletter
\let\realnormalsize=\normalsize
\global\topskip=2sp
\def\preveqno{}\let\real@float=\@float \let\realend@float=\end@float
\def\@float{\let\@savefreelist\@freelist\real@float}
\def\liih@math{\ifmmode$\else\bad@math\fi}
\def\end@float{\realend@float\global\let\@freelist\@savefreelist}
\let\real@dbflt=\@dbflt \let\end@dblfloat=\end@float
\let\@largefloatcheck=\relax
\let\if@boxedmulticols=\iftrue
\def\@dbflt{\let\@savefreelist\@freelist\real@dbflt}
\def\adjustnormalsize{\def\normalsize{\mathsurround=0pt \realnormalsize
 \parindent=0pt\abovedisplayskip=0pt\belowdisplayskip=0pt}%
 \def\phantompar{\csname par\endcsname}\normalsize}%
\def\lthtmltypeout#1{{\let\protect\string \immediate\write\lthtmlwrite{#1}}}%
\newcommand\lthtmlhboxmathA{\adjustnormalsize\setbox\sizebox=\hbox\bgroup\kern.05em }%
\newcommand\lthtmlhboxmathB{\adjustnormalsize\setbox\sizebox=\hbox to\hsize\bgroup\hfill }%
\newcommand\lthtmlvboxmathA{\adjustnormalsize\setbox\sizebox=\vbox\bgroup %
 \let\ifinner=\iffalse \let\)\liih@math }%
\newcommand\lthtmlboxmathZ{\@next\next\@currlist{}{\def\next{\voidb@x}}%
 \expandafter\box\next\egroup}%
\newcommand\lthtmlmathtype[1]{\gdef\lthtmlmathenv{#1}}%
\newcommand\lthtmllogmath{\dimen0\ht\sizebox \advance\dimen0\dp\sizebox
  \ifdim\dimen0>.95\vsize
   \lthtmltypeout{%
*** image for \lthtmlmathenv\space is too tall at \the\dimen0, reducing to .95 vsize ***}%
   \ht\sizebox.95\vsize \dp\sizebox\z@ \fi
  \lthtmltypeout{l2hSize %
:\lthtmlmathenv:\the\ht\sizebox::\the\dp\sizebox::\the\wd\sizebox.\preveqno}}%
\newcommand\lthtmlfigureA[1]{\let\@savefreelist\@freelist
       \lthtmlmathtype{#1}\lthtmlvboxmathA}%
\newcommand\lthtmlpictureA{\bgroup\catcode`\_=8 \lthtmlpictureB}%
\newcommand\lthtmlpictureB[1]{\lthtmlmathtype{#1}\egroup
       \let\@savefreelist\@freelist \lthtmlhboxmathB}%
\newcommand\lthtmlpictureZ[1]{\hfill\lthtmlfigureZ}%
\newcommand\lthtmlfigureZ{\lthtmlboxmathZ\lthtmllogmath\copy\sizebox
       \global\let\@freelist\@savefreelist}%
\newcommand\lthtmldisplayA{\bgroup\catcode`\_=8 \lthtmldisplayAi}%
\newcommand\lthtmldisplayAi[1]{\lthtmlmathtype{#1}\egroup\lthtmlvboxmathA}%
\newcommand\lthtmldisplayB[1]{\edef\preveqno{(\theequation)}%
  \lthtmldisplayA{#1}\let\@eqnnum\relax}%
\newcommand\lthtmldisplayZ{\lthtmlboxmathZ\lthtmllogmath\lthtmlsetmath}%
\newcommand\lthtmlinlinemathA{\bgroup\catcode`\_=8 \lthtmlinlinemathB}
\newcommand\lthtmlinlinemathB[1]{\lthtmlmathtype{#1}\egroup\lthtmlhboxmathA
  \vrule height1.5ex width0pt }%
\newcommand\lthtmlinlineA{\bgroup\catcode`\_=8 \lthtmlinlineB}%
\newcommand\lthtmlinlineB[1]{\lthtmlmathtype{#1}\egroup\lthtmlhboxmathA}%
\newcommand\lthtmlinlineZ{\egroup\expandafter\ifdim\dp\sizebox>0pt %
  \expandafter\centerinlinemath\fi\lthtmllogmath\lthtmlsetinline}
\newcommand\lthtmlinlinemathZ{\egroup\expandafter\ifdim\dp\sizebox>0pt %
  \expandafter\centerinlinemath\fi\lthtmllogmath\lthtmlsetmath}
\newcommand\lthtmlindisplaymathZ{\egroup %
  \centerinlinemath\lthtmllogmath\lthtmlsetmath}
\def\lthtmlsetinline{\hbox{\vrule width.1em \vtop{\vbox{%
  \kern.1em\copy\sizebox}\ifdim\dp\sizebox>0pt\kern.1em\else\kern.3pt\fi
  \ifdim\hsize>\wd\sizebox \hrule depth1pt\fi}}}
\def\lthtmlsetmath{\hbox{\vrule width.1em\kern-.05em\vtop{\vbox{%
  \kern.1em\kern0.8 pt\hbox{\hglue.17em\copy\sizebox\hglue0.8 pt}}\kern.3pt%
  \ifdim\dp\sizebox>0pt\kern.1em\fi \kern0.8 pt%
  \ifdim\hsize>\wd\sizebox \hrule depth1pt\fi}}}
\def\centerinlinemath{%
  \dimen1=\ifdim\ht\sizebox<\dp\sizebox \dp\sizebox\else\ht\sizebox\fi
  \advance\dimen1by.5pt \vrule width0pt height\dimen1 depth\dimen1 
 \dp\sizebox=\dimen1\ht\sizebox=\dimen1\relax}

\def\lthtmlcheckvsize{\ifdim\ht\sizebox<\vsize 
  \ifdim\wd\sizebox<\hsize\expandafter\hfill\fi \expandafter\vfill
  \else\expandafter\vss\fi}%
\providecommand{\selectlanguage}[1]{}%
\makeatletter \tracingstats = 1 
\providecommand{\Beta}{\textrm{B}}
\providecommand{\Mu}{\textrm{M}}
\providecommand{\Kappa}{\textrm{K}}
\providecommand{\Rho}{\textrm{R}}
\providecommand{\Epsilon}{\textrm{E}}
\providecommand{\Chi}{\textrm{X}}
\providecommand{\Iota}{\textrm{J}}
\providecommand{\omicron}{\textrm{o}}
\providecommand{\Zeta}{\textrm{Z}}
\providecommand{\Eta}{\textrm{H}}
\providecommand{\Omicron}{\textrm{O}}
\providecommand{\Nu}{\textrm{N}}
\providecommand{\Tau}{\textrm{T}}
\providecommand{\Alpha}{\textrm{A}}


\begin{document}
\pagestyle{empty}\thispagestyle{empty}\lthtmltypeout{}%
\lthtmltypeout{latex2htmlLength hsize=\the\hsize}\lthtmltypeout{}%
\lthtmltypeout{latex2htmlLength vsize=\the\vsize}\lthtmltypeout{}%
\lthtmltypeout{latex2htmlLength hoffset=\the\hoffset}\lthtmltypeout{}%
\lthtmltypeout{latex2htmlLength voffset=\the\voffset}\lthtmltypeout{}%
\lthtmltypeout{latex2htmlLength topmargin=\the\topmargin}\lthtmltypeout{}%
\lthtmltypeout{latex2htmlLength topskip=\the\topskip}\lthtmltypeout{}%
\lthtmltypeout{latex2htmlLength headheight=\the\headheight}\lthtmltypeout{}%
\lthtmltypeout{latex2htmlLength headsep=\the\headsep}\lthtmltypeout{}%
\lthtmltypeout{latex2htmlLength parskip=\the\parskip}\lthtmltypeout{}%
\lthtmltypeout{latex2htmlLength oddsidemargin=\the\oddsidemargin}\lthtmltypeout{}%
\makeatletter
\if@twoside\lthtmltypeout{latex2htmlLength evensidemargin=\the\evensidemargin}%
\else\lthtmltypeout{latex2htmlLength evensidemargin=\the\oddsidemargin}\fi%
\lthtmltypeout{}%
\makeatother
\setcounter{page}{1}
\onecolumn

% !!! IMAGES START HERE !!!

{\newpage\clearpage
\lthtmlinlinemathA{tex2html_wrap_inline1113}%
$ X = \{ 1, 2, 3, 4 \} , R = \{ \{1, 2, 3\}, \{1, 3\} \{ 4 \} \}$%
\lthtmlinlinemathZ
\lthtmlcheckvsize\clearpage}

{\newpage\clearpage
\lthtmlinlinemathA{tex2html_wrap_inline1115}%
$ X = \{A, B, C\}, R = \{ \{ A, B \}, \{ A \}, \{ A, B \}, \{ C \} \}$%
\lthtmlinlinemathZ
\lthtmlcheckvsize\clearpage}

{\newpage\clearpage
\lthtmlinlinemathA{tex2html_wrap_inline1125}%
$ X = \{ 1, 2, 3, 4, 5, 6, 7, 8 \}$%
\lthtmlinlinemathZ
\lthtmlcheckvsize\clearpage}

{\newpage\clearpage
\lthtmlinlinemathA{tex2html_wrap_inline1127}%
$ R = \{ A:\{1, 2, 3\}, B:\{2, 3, 4, 5\}, C:\{4, 5, 6\}, D:\{3, 5, 7\}, E:\{7, 8\} \}$%
\lthtmlinlinemathZ
\lthtmlcheckvsize\clearpage}

{\newpage\clearpage
\lthtmlinlinemathA{tex2html_wrap_inline1129}%
$ X = \{ A, B, C, D, E \}$%
\lthtmlinlinemathZ
\lthtmlcheckvsize\clearpage}

{\newpage\clearpage
\lthtmlinlinemathA{tex2html_wrap_inline1131}%
$ R = \{ 1:\{A\}, 2:\{A, B\}, 3:\{A, B, D\}, 4:\{B, C\}, 5:\{B, C, D\}, 6:\{C\}, 7:\{D, E\}, 8:\{E\} \}$%
\lthtmlinlinemathZ
\lthtmlcheckvsize\clearpage}

{\newpage\clearpage
\lthtmlpictureA{tex2html_wrap750}%
\framebox{
\vbox{
\hbox to 5.78in { {\bf COMPSCI 634: Geometric Algorithms} \hfill 4/3/2014 }
\vspace{4mm}
\hbox to 5.78in { {\Large \hfill Lecture \#24: Geometric Set Cover  \hfill} }
\vspace{2mm}
\hbox to 5.78in { {\it Lecturer: Pankaj Agarwal \hfill Scribe: Chris Tralie} }
}
}%
\lthtmlpictureZ
\lthtmlcheckvsize\clearpage}

\stepcounter{section}
{\newpage\clearpage
\lthtmlinlinemathA{tex2html_wrap_inline1135}%
$ \epsilon$%
\lthtmlinlinemathZ
\lthtmlcheckvsize\clearpage}

\stepcounter{section}
\stepcounter{subsection}
{\newpage\clearpage
\lthtmlinlinemathA{tex2html_wrap_inline1148}%
$ \Sigma = (X, R)$%
\lthtmlinlinemathZ
\lthtmlcheckvsize\clearpage}

{\newpage\clearpage
\lthtmlinlinemathA{tex2html_wrap_inline1150}%
$ X$%
\lthtmlinlinemathZ
\lthtmlcheckvsize\clearpage}

{\newpage\clearpage
\lthtmlinlinemathA{tex2html_wrap_indisplay1152}%
$\displaystyle \Sigma^T = \left( R, \{ \{ r_j | x_i \in r_j \} | x_i \in X \} \right) $%
\lthtmlindisplaymathZ
\lthtmlcheckvsize\clearpage}

{\newpage\clearpage
\lthtmlinlinemathA{tex2html_wrap_inline1154}%
$ \Sigma$%
\lthtmlinlinemathZ
\lthtmlcheckvsize\clearpage}

{\newpage\clearpage
\lthtmlinlinemathA{tex2html_wrap_inline1156}%
$ x \in X$%
\lthtmlinlinemathZ
\lthtmlcheckvsize\clearpage}

{\newpage\clearpage
\lthtmlinlinemathA{tex2html_wrap_inline1162}%
$ r \in R$%
\lthtmlinlinemathZ
\lthtmlcheckvsize\clearpage}

{\newpage\clearpage
\lthtmlinlinemathA{tex2html_wrap_inline1164}%
$ x \in r$%
\lthtmlinlinemathZ
\lthtmlcheckvsize\clearpage}

{\newpage\clearpage
\lthtmlinlinemathA{tex2html_wrap_inline1177}%
$ A$%
\lthtmlinlinemathZ
\lthtmlcheckvsize\clearpage}

{\newpage\clearpage
\lthtmlinlinemathA{tex2html_wrap_inline1179}%
$ A^T$%
\lthtmlinlinemathZ
\lthtmlcheckvsize\clearpage}

\stepcounter{subsection}
{\newpage\clearpage
\lthtmlinlinemathA{tex2html_wrap_inline1184}%
$ \mathbb{R}^2$%
\lthtmlinlinemathZ
\lthtmlcheckvsize\clearpage}

{\newpage\clearpage
\lthtmlinlinemathA{tex2html_wrap_inline1186}%
$ \gamma$%
\lthtmlinlinemathZ
\lthtmlcheckvsize\clearpage}

{\newpage\clearpage
\lthtmlinlinemathA{tex2html_wrap_inline1188}%
$ m$%
\lthtmlinlinemathZ
\lthtmlcheckvsize\clearpage}

{\newpage\clearpage
\lthtmlinlinemathA{tex2html_wrap_inline1190}%
$ (\gamma_1, ..., \gamma_m)$%
\lthtmlinlinemathZ
\lthtmlcheckvsize\clearpage}

{\newpage\clearpage
\lthtmlinlinemathA{tex2html_wrap_inline1192}%
$ \Sigma: (X, \{ \gamma \cap X | \gamma \in \Gamma \} ) $%
\lthtmlinlinemathZ
\lthtmlcheckvsize\clearpage}

{\newpage\clearpage
\lthtmlinlinemathA{tex2html_wrap_inline1194}%
$ \Sigma^T: ( \Gamma, \{ \{ \gamma | x \in \gamma\} | x \in X \} )$%
\lthtmlinlinemathZ
\lthtmlcheckvsize\clearpage}

\stepcounter{section}
\stepcounter{subsection}
{\newpage\clearpage
\lthtmlinlinemathA{tex2html_wrap_inline1224}%
$ H \subset X$%
\lthtmlinlinemathZ
\lthtmlcheckvsize\clearpage}

{\newpage\clearpage
\lthtmlinlinemathA{tex2html_wrap_indisplay1228}%
$\displaystyle H \cap r \neq \emptyset \forall r \in R $%
\lthtmlindisplaymathZ
\lthtmlcheckvsize\clearpage}

{\newpage\clearpage
\lthtmlinlinemathA{tex2html_wrap_inline1237}%
$ S \subset R$%
\lthtmlinlinemathZ
\lthtmlcheckvsize\clearpage}

{\newpage\clearpage
\lthtmlinlinemathA{tex2html_wrap_indisplay1241}%
$\displaystyle \cup_{s \in S} s = X $%
\lthtmlindisplaymathZ
\lthtmlcheckvsize\clearpage}

{\newpage\clearpage
\lthtmlinlinemathA{tex2html_wrap_inline1245}%
$ \Sigma^T$%
\lthtmlinlinemathZ
\lthtmlcheckvsize\clearpage}

{\newpage\clearpage
\lthtmlinlinemathA{tex2html_wrap_inline1247}%
$ \epsilon-$%
\lthtmlinlinemathZ
\lthtmlcheckvsize\clearpage}

{\newpage\clearpage
\lthtmlinlinemathA{tex2html_wrap_inline1256}%
$ N \subset X$%
\lthtmlinlinemathZ
\lthtmlcheckvsize\clearpage}

{\newpage\clearpage
\lthtmlinlinemathA{tex2html_wrap_inline1262}%
$ \forall r \in R$%
\lthtmlinlinemathZ
\lthtmlcheckvsize\clearpage}

{\newpage\clearpage
\lthtmlinlinemathA{tex2html_wrap_indisplay1264}%
$\displaystyle |r| \geq \epsilon |x| \implies N \cap r \neq \emptyset $%
\lthtmlindisplaymathZ
\lthtmlcheckvsize\clearpage}

{\newpage\clearpage
\lthtmlinlinemathA{tex2html_wrap_inline1266}%
$ \epsilon = \frac{1}{N}$%
\lthtmlinlinemathZ
\lthtmlcheckvsize\clearpage}

{\newpage\clearpage
\lthtmlinlinemathA{tex2html_wrap_inline1270}%
$ N = |X|$%
\lthtmlinlinemathZ
\lthtmlcheckvsize\clearpage}

{\newpage\clearpage
\lthtmlinlinemathA{tex2html_wrap_inline1278}%
$ N$%
\lthtmlinlinemathZ
\lthtmlcheckvsize\clearpage}

{\newpage\clearpage
\lthtmlinlinemathA{tex2html_wrap_inline1291}%
$ d$%
\lthtmlinlinemathZ
\lthtmlcheckvsize\clearpage}

{\newpage\clearpage
\lthtmlinlinemathA{tex2html_wrap_inline1293}%
$ \delta, \epsilon > 0$%
\lthtmlinlinemathZ
\lthtmlcheckvsize\clearpage}

{\newpage\clearpage
\lthtmlinlinemathA{tex2html_wrap_indisplay1297}%
$\displaystyle O\left( \frac{d}{\epsilon} \log \frac{1}{\delta \epsilon} \right) $%
\lthtmlindisplaymathZ
\lthtmlcheckvsize\clearpage}

{\newpage\clearpage
\lthtmlinlinemathA{tex2html_wrap_inline1303}%
$ \geq (1 - \delta)$%
\lthtmlinlinemathZ
\lthtmlcheckvsize\clearpage}

{\newpage\clearpage
\lthtmlinlinemathA{tex2html_wrap_inline1305}%
$ O(\frac{1}{\epsilon} \log \log \frac{1}{\epsilon} )$%
\lthtmlinlinemathZ
\lthtmlcheckvsize\clearpage}

{\newpage\clearpage
\lthtmlinlinemathA{tex2html_wrap_inline1309}%
$ O(\frac{1}{\epsilon} \log \frac{1}{\epsilon} )$%
\lthtmlinlinemathZ
\lthtmlcheckvsize\clearpage}

\stepcounter{subsection}
{\newpage\clearpage
\lthtmlinlinemathA{tex2html_wrap_inline1320}%
$ \frac{1}{N}$%
\lthtmlinlinemathZ
\lthtmlcheckvsize\clearpage}

{\newpage\clearpage
\lthtmlinlinemathA{tex2html_wrap_indisplay1331}%
$\displaystyle w: X \rightarrow \mathbb{Z}^+$%
\lthtmlindisplaymathZ
\lthtmlcheckvsize\clearpage}

{\newpage\clearpage
\lthtmlinlinemathA{tex2html_wrap_inline1333}%
$ S \in R$%
\lthtmlinlinemathZ
\lthtmlcheckvsize\clearpage}

{\newpage\clearpage
\lthtmlinlinemathA{tex2html_wrap_indisplay1335}%
$\displaystyle w(S) = \sum_{x \in S} w(x) $%
\lthtmlindisplaymathZ
\lthtmlcheckvsize\clearpage}

{\newpage\clearpage
\lthtmlinlinemathA{tex2html_wrap_inline1341}%
$ (\Sigma = (X, R), w)$%
\lthtmlinlinemathZ
\lthtmlcheckvsize\clearpage}

{\newpage\clearpage
\lthtmlinlinemathA{tex2html_wrap_indisplay1345}%
$\displaystyle w(r) \geq \epsilon w(x) \implies r \cap N \neq \emptyset $%
\lthtmlindisplaymathZ
\lthtmlcheckvsize\clearpage}

{\newpage\clearpage
\lthtmlinlinemathA{tex2html_wrap_inline1347}%
$ r$%
\lthtmlinlinemathZ
\lthtmlcheckvsize\clearpage}

{\newpage\clearpage
\lthtmlinlinemathA{tex2html_wrap_inline1351}%
$ w(r) < \epsilon w(x)$%
\lthtmlinlinemathZ
\lthtmlcheckvsize\clearpage}

{\newpage\clearpage
\lthtmlinlinemathA{tex2html_wrap_inline1353}%
$ w$%
\lthtmlinlinemathZ
\lthtmlcheckvsize\clearpage}

{\newpage\clearpage
\lthtmlinlinemathA{tex2html_wrap_inline890}%

% latex2html id marker 890
\fbox{
{\em
\begin{minipage}{0.5\textwidth}
\par
\begin{algorithmic}
\State ALGORITHM HittingSetApprox1$(\Sigma = (X, R) )$
\State Initially, $w(x) = 1 \forall x \in X$
\par
\While{$\exists$\  an $\epsilon$-light range}
\par
\State Choose an $\epsilon$-light range $r \in R$
\par
\State $w(x) \leftarrow 2 w(x) \forall x \in r$
\par
\EndWhile
\par
\State \Return $\epsilon$-net of $(\Sigma, w)$
\end{algorithmic}
\par
\end{minipage}
}
}%
\lthtmlinlinemathZ
\lthtmlcheckvsize\clearpage}

{\newpage\clearpage
\lthtmlinlinemathA{tex2html_wrap_inline1365}%
$ w_i(x)$%
\lthtmlinlinemathZ
\lthtmlcheckvsize\clearpage}

{\newpage\clearpage
\lthtmlinlinemathA{tex2html_wrap_inline1367}%
$ w(x)$%
\lthtmlinlinemathZ
\lthtmlcheckvsize\clearpage}

{\newpage\clearpage
\lthtmlinlinemathA{tex2html_wrap_inline1369}%
$ i$%
\lthtmlinlinemathZ
\lthtmlcheckvsize\clearpage}

{\newpage\clearpage
\lthtmlinlinemathA{tex2html_wrap_inline1373}%
$ H^*$%
\lthtmlinlinemathZ
\lthtmlcheckvsize\clearpage}

{\newpage\clearpage
\lthtmlinlinemathA{tex2html_wrap_inline1375}%
$ k$%
\lthtmlinlinemathZ
\lthtmlcheckvsize\clearpage}

{\newpage\clearpage
\lthtmlinlinemathA{tex2html_wrap_inline1381}%
$ w_i(r) < \epsilon$%
\lthtmlinlinemathZ
\lthtmlcheckvsize\clearpage}

{\newpage\clearpage
\lthtmlinlinemathA{tex2html_wrap_indisplay1383}%
$\displaystyle w_{i+1}(X) = w_i(X) + w_i(r) \leq (1 + \epsilon) w_i(X) $%
\lthtmlindisplaymathZ
\lthtmlcheckvsize\clearpage}

{\newpage\clearpage
\lthtmlinlinemathA{tex2html_wrap_inline1385}%
$ w_0(X) = n$%
\lthtmlinlinemathZ
\lthtmlcheckvsize\clearpage}

{\newpage\clearpage
\lthtmlinlinemathA{tex2html_wrap_indisplay1387}%
$\displaystyle w_{i}(X) \leq n(1+\epsilon)^i $%
\lthtmlindisplaymathZ
\lthtmlcheckvsize\clearpage}

{\newpage\clearpage
\lthtmlinlinemathA{tex2html_wrap_inline1389}%
$ w(H^*)$%
\lthtmlinlinemathZ
\lthtmlcheckvsize\clearpage}

{\newpage\clearpage
\lthtmlinlinemathA{tex2html_wrap_indisplay1395}%
$\displaystyle w(H^*) \geq k + i $%
\lthtmlindisplaymathZ
\lthtmlcheckvsize\clearpage}

{\newpage\clearpage
\lthtmlinlinemathA{tex2html_wrap_inline1397}%
$ f(i) = k + i$%
\lthtmlinlinemathZ
\lthtmlcheckvsize\clearpage}

{\newpage\clearpage
\lthtmlinlinemathA{tex2html_wrap_inline1399}%
$ g(i) = k2^{i/k}$%
\lthtmlinlinemathZ
\lthtmlcheckvsize\clearpage}

{\newpage\clearpage
\lthtmlinlinemathA{tex2html_wrap_inline1401}%
$ f(i) > g(i)$%
\lthtmlinlinemathZ
\lthtmlcheckvsize\clearpage}

{\newpage\clearpage
\lthtmlinlinemathA{tex2html_wrap_inline1403}%
$ [0, k]$%
\lthtmlinlinemathZ
\lthtmlcheckvsize\clearpage}

{\newpage\clearpage
\lthtmlinlinemathA{tex2html_wrap_inline1405}%
$ f(0) = g(0)$%
\lthtmlinlinemathZ
\lthtmlcheckvsize\clearpage}

{\newpage\clearpage
\lthtmlinlinemathA{tex2html_wrap_inline1407}%
$ f(k) = g(k)$%
\lthtmlinlinemathZ
\lthtmlcheckvsize\clearpage}

{\newpage\clearpage
\lthtmlinlinemathA{tex2html_wrap_inline1409}%
$ f'(0) > g'(0)$%
\lthtmlinlinemathZ
\lthtmlcheckvsize\clearpage}

{\newpage\clearpage
\lthtmlinlinemathA{tex2html_wrap_inline1415}%
$ i > k$%
\lthtmlinlinemathZ
\lthtmlcheckvsize\clearpage}

{\newpage\clearpage
\lthtmlinlinemathA{tex2html_wrap_indisplay1419}%
$\displaystyle w(H^*) \geq k 2^{i/k} $%
\lthtmlindisplaymathZ
\lthtmlcheckvsize\clearpage}

{\newpage\clearpage
\lthtmlinlinemathA{tex2html_wrap_inline1421}%
$ k = |H^*|$%
\lthtmlinlinemathZ
\lthtmlcheckvsize\clearpage}

{\newpage\clearpage
\lthtmlinlinemathA{tex2html_wrap_indisplay1423}%
$\displaystyle \epsilon = \frac{\ln \sqrt{2}}{k} $%
\lthtmlindisplaymathZ
\lthtmlcheckvsize\clearpage}

{\newpage\clearpage
\lthtmlinlinemathA{tex2html_wrap_indisplay1425}%
$\displaystyle w_i(X) \leq (1 + \epsilon)^i n = \left(1 + \frac{\ln \sqrt{2}}{k} \right)^i n \leq  \exp\left(  i \frac{\ln \sqrt{2}}{k} \right) n$%
\lthtmlindisplaymathZ
\lthtmlcheckvsize\clearpage}

{\newpage\clearpage
\lthtmlinlinemathA{tex2html_wrap_inline1427}%
$ H^* \subset X$%
\lthtmlinlinemathZ
\lthtmlcheckvsize\clearpage}

{\newpage\clearpage
\lthtmlinlinemathA{tex2html_wrap_indisplay1429}%
$\displaystyle w_i(H^*) \leq w_i(X) \leq \exp\left(  i \frac{\ln \sqrt{2}}{k} \right) n $%
\lthtmlindisplaymathZ
\lthtmlcheckvsize\clearpage}

{\newpage\clearpage
\lthtmlinlinemathA{tex2html_wrap_inline1431}%
$ w_i(H^*)$%
\lthtmlinlinemathZ
\lthtmlcheckvsize\clearpage}

{\newpage\clearpage
\lthtmlinlinemathA{tex2html_wrap_indisplay1433}%
$\displaystyle k 2^{i/k} \leq n \exp \left( i \frac{\ln \sqrt{2}}{k} \right) $%
\lthtmlindisplaymathZ
\lthtmlcheckvsize\clearpage}

{\newpage\clearpage
\lthtmlinlinemathA{tex2html_wrap_indisplay1435}%
$\displaystyle \ln(k) + \ln(2) \frac{i}{k} \leq \ln(n) + \ln \sqrt{2} \frac{i}{k} $%
\lthtmlindisplaymathZ
\lthtmlcheckvsize\clearpage}

{\newpage\clearpage
\lthtmlinlinemathA{tex2html_wrap_inline1437}%
$ \epsilon = \frac{\ln \sqrt{2}}{k}$%
\lthtmlinlinemathZ
\lthtmlcheckvsize\clearpage}

{\newpage\clearpage
\lthtmlinlinemathA{tex2html_wrap_inline1439}%
$ \ln \sqrt{2} \frac{i}{k}$%
\lthtmlinlinemathZ
\lthtmlcheckvsize\clearpage}

{\newpage\clearpage
\lthtmlinlinemathA{tex2html_wrap_inline1441}%
$ \frac{i}{k}$%
\lthtmlinlinemathZ
\lthtmlcheckvsize\clearpage}

{\newpage\clearpage
\lthtmlinlinemathA{tex2html_wrap_indisplay1443}%
$\displaystyle \frac{i}{k} \ln(\sqrt{2}) \leq \ln \left( \frac{n}{k} \right) $%
\lthtmlindisplaymathZ
\lthtmlcheckvsize\clearpage}

{\newpage\clearpage
\lthtmlinlinemathA{tex2html_wrap_indisplay1445}%
$\displaystyle i = O\left( k \log \frac{n}{k} \right) $%
\lthtmlindisplaymathZ
\lthtmlcheckvsize\clearpage}

{\newpage\clearpage
\lthtmlinlinemathA{tex2html_wrap_inline1455}%
$ (k/\sqrt{2}) \log\left(\frac{n}{k}\right)$%
\lthtmlinlinemathZ
\lthtmlcheckvsize\clearpage}

{\newpage\clearpage
\lthtmlinlinemathA{tex2html_wrap_inline1459}%
$ (\Sigma, w)$%
\lthtmlinlinemathZ
\lthtmlcheckvsize\clearpage}

{\newpage\clearpage
\lthtmlinlinemathA{tex2html_wrap_inline1461}%
$ O\left( k \log \frac{n}{k} \right)$%
\lthtmlinlinemathZ
\lthtmlcheckvsize\clearpage}

{\newpage\clearpage
\lthtmlinlinemathA{tex2html_wrap_inline1480}%
$ \Theta(\frac{1}{\epsilon})$%
\lthtmlinlinemathZ
\lthtmlcheckvsize\clearpage}

\stepcounter{subsection}
{\newpage\clearpage
\lthtmlinlinemathA{tex2html_wrap_inline950}%

% latex2html id marker 950
\fbox{
{\em
\begin{minipage}{0.8\textwidth}
\par
\begin{algorithmic}
\State ALGORITHM HittingSetApprox2$(\Sigma = (X, R) )$
\par
\State
\par
For the range space $\Sigma = (X, R)$
\begin{flalign}
 &\text{Let} \mu = \frac{2}{\ln 2} k \ln |R|^2|X|&
\end{flalign}
\State Initially, $w(x) = 1 \forall x \in X$
\State Initially, $w(r) = 1 \forall r \in R$
\par
\For{$i = 1$\  to $\mu$}
\par
\State 
	\begin{itemize}
	\item Sample a random $\overline{x_i} \in X$\  with the probability distribution $\text{Pr}(x) = w(x)/w(X)$ 
\par
\item Sample a random range $\overline{r_i} \in R$\  with the probability distribution $\text{Pr}(r) = w(r)/w(R)$
\par
\end{itemize}
\par
\State $\forall r \text{ s.t. } x_i \in r, w(r) \leftarrow w(r)/2 $
\par
\State $\forall x \in r, w(x) = 2w(x)$
\par
\EndFor
\end{algorithmic}
\par
\end{minipage}
}
}%
\lthtmlinlinemathZ
\lthtmlcheckvsize\clearpage}

{\newpage\clearpage
\lthtmlinlinemathA{tex2html_wrap_inline1489}%
$ \Pi(x_i)$%
\lthtmlinlinemathZ
\lthtmlcheckvsize\clearpage}

{\newpage\clearpage
\lthtmlinlinemathA{tex2html_wrap_inline1491}%
$ k \leq \mu$%
\lthtmlinlinemathZ
\lthtmlcheckvsize\clearpage}

{\newpage\clearpage
\lthtmlinlinemathA{tex2html_wrap_inline1493}%
$ \overline{x_k}  = x_i$%
\lthtmlinlinemathZ
\lthtmlcheckvsize\clearpage}

{\newpage\clearpage
\lthtmlinlinemathA{tex2html_wrap_inline1495}%
$ 1/(8k)$%
\lthtmlinlinemathZ
\lthtmlcheckvsize\clearpage}

{\newpage\clearpage
\lthtmlinlinemathA{tex2html_wrap_inline1497}%
$ (\Sigma, \Pi)$%
\lthtmlinlinemathZ
\lthtmlcheckvsize\clearpage}

{\newpage\clearpage
\lthtmlinlinemathA{tex2html_wrap_inline1499}%
$ O(1)$%
\lthtmlinlinemathZ
\lthtmlcheckvsize\clearpage}

\stepcounter{section}
{\newpage\clearpage
\lthtmlinlinemathA{tex2html_wrap_inline1502}%
$ n/2$%
\lthtmlinlinemathZ
\lthtmlcheckvsize\clearpage}

{\newpage\clearpage
\lthtmlinlinemathA{tex2html_wrap_inline1504}%
$ \log^2 n$%
\lthtmlinlinemathZ
\lthtmlcheckvsize\clearpage}

{\newpage\clearpage
\lthtmlinlinemathA{tex2html_wrap_inline1509}%
$ R$%
\lthtmlinlinemathZ
\lthtmlcheckvsize\clearpage}

{\newpage\clearpage
\lthtmlinlinemathA{tex2html_wrap_inline1513}%
$ \forall r_1, r_2 \in S, r_1 \cap r_2 = \emptyset$%
\lthtmlinlinemathZ
\lthtmlcheckvsize\clearpage}

{\newpage\clearpage
\lthtmlinlinemathA{tex2html_wrap_inline1515}%
$ \log n$%
\lthtmlinlinemathZ
\lthtmlcheckvsize\clearpage}

{\newpage\clearpage
\lthtmlinlinemathA{tex2html_wrap_inline972}%

% latex2html id marker 972
\fbox{
{\em
\begin{minipage}{0.8\textwidth}
\par
\begin{algorithmic}
\State ALGORITHM 2DIS(R)
\par
\State
Draw a vertical line $l$\  s.t. each side contains an equal number of vertices on the rectangles $R$
\par
\State Define the sets
\par
\begin{itemize}
\par
\item $R_0 = \{ r | l \cap r \neq \emptyset \}$
\item $R^- = \{ r | r \text{  lies to the left of } l \}$
\item $R^+ = \{ r | r \text{  lies to the left of } l \}$
\par
\end{itemize}
\par
\State Determine the independent set $I(R_0)$\  with a 1D greedy algorithm.
\par
\State Return $I(R) = I(R_0) \cup 2DIS(R^-) \cup 2DIS(R^+)$
\par
\end{algorithmic}
\par
\end{minipage}
}
}%
\lthtmlinlinemathZ
\lthtmlcheckvsize\clearpage}

{\newpage\clearpage
\lthtmlinlinemathA{tex2html_wrap_inline1528}%
$ O(\log\log n)$%
\lthtmlinlinemathZ
\lthtmlcheckvsize\clearpage}

{\newpage\clearpage
\lthtmlinlinemathA{tex2html_wrap_inline1555}%
$ \backslash$%
\lthtmlinlinemathZ
\lthtmlcheckvsize\clearpage}


\end{document}
